   

\usepackage{amsmath,amsthm,amsfonts,amssymb,titlesec}
\usepackage{hyperref}
\usepackage{tikz}
\usepackage{verbatim}
\usepackage{accents}
\usepackage[citestyle=alphabetic,bibstyle=alphabetic,backend=bibtex]{biblatex}
\usepackage{todonotes}
\usepackage[american]{babel}
\usepackage{fancyhdr}

%\renewbibmacro{in:}{}

\hypersetup{colorlinks=false}
\usetikzlibrary{calc, decorations.pathreplacing,shapes.misc}
\usetikzlibrary{decorations.pathmorphing}
\usepackage[left=1in,top=1in,right=1in]{geometry}
\usepackage[capitalize]{cleveref}




\newcommand{\mathcolorbox}[2]{\colorbox{#1}{$\displaystyle #2$}}
\newcommand{\xxx}{T base with combinatorial potential data }
\newcommand{\Xxx}{T base with combinatorial potential data }
\newcommand{\xxxc}{combinatorial potential stratified space }
\newcommand{\Xxxc}{combinatorial potential stratified space }
\newcommand{\argument}{symplectic character }
\newcommand{\arguments}{symplectic characters }
\newcommand{\snip}[2]{#1}


\newtheorem{theorem}{Theorem}[section]
\newtheorem{lem}[theorem]{Lemma}
\newtheorem{lemma}[theorem]{Lemma}
\newtheorem{proposition}[theorem]{Proposition}
\newtheorem{corollary}[theorem]{Corollary}
\newtheorem{conjecture}[theorem]{Conjecture}
\newtheorem{notation}[theorem]{Notation}
\newtheorem{question}[theorem]{Question}


\theoremstyle{remark} 
\newtheorem{rem}[theorem]{Remark}
\newtheorem{remark}[theorem]{Remark}
 \crefname{rem}{Remark}{Remarks}
 \Crefname{rem}{Remark}{Remarks}
 \newtheorem{exercise}[theorem]{Exercise}
 \newtheorem{example}[theorem]{Example}
 \newenvironment{construction}{}{}
 \newenvironment{exposition}{}{}
 \newenvironment{application}{}{}

\theoremstyle{definition} 
\newtheorem{df}[theorem]{Definition} 
\newtheorem{definition}[theorem]{Definition} 

\titleformat*{\section}{\normalsize \bfseries \filcenter}
\titleformat*{\subsection}{\normalsize \bfseries }


\newtheorem{mainthm}{Theorem}
\Crefname{mainthm}{Theorem}{Theorems}
\newtheorem{maincor}[mainthm]{Corollary}
\Crefname{maincor}{Corollary}{Corollaries}


\renewcommand*{\themainthm}{\Alph{mainthm}}


\makeatletter
\def\namedlabel#1#2{\begingroup
   \def\@currentlabel{#2}%
   \label{#1}\endgroup
}
\makeatother


